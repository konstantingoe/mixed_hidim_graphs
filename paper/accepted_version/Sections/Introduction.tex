Graphical models are widely used in the analysis of multivariate data, providing a convenient and interpretable way to study relationships among potentially large numbers of variables. They are key tools in modern statistics and machine learning and play an important role in diverse applications. Undirected graphical models are used in a wide range of settings, including, among others, systems biology, omics, deep phenotyping \cite[see, e.g.][]{dobra2004, Finegold11, monti2014}
and as a component within other analyses, including two-sample testing, unsupervised learning, hidden Markov modeling, and more \cite[examples include][]{wei2007,verzelen2009,stadler2013, stadler2015,perrakis2021}.

A significant portion of the literature on graphical models has concentrated on scenarios where either only continuous variables or only discrete variables are present. Regarding the former case, Gaussian graphical models have been extensively studied, including in the high-dimensional regime \cite[see among others][]{Meinshausen06, Friedman08, Banerjee08, Lam09, Yuan10, Ravikumar11, Cai11}. In such models, it is assumed that the observed random vector follows a multivariate Gaussian distribution, and the graph structure of the model is given by the zero pattern in the inverse covariance matrix. Generalizations for continuous, non-Gaussian data have also been studied \cite{Miyamura06, Liu09, Finegold11}. In the latter case, discrete graphical models -- related to Ising-type models in statistical physics -- have also been extensively studied \cite[see, e.g.][]{wainwright2006, ravikumar2010}.

However, in many applications, it is common to encounter data that entail \textit{mixed} variable types, i.e., where the data vector includes components of different types (e.g., continuous-Gaussian, continuous-non-Gaussian, count, binary, etc.). Such ``column heterogeneity" (from the usual convention of samples in rows and variables in columns) is the rule rather than the exception. For instance, in biomedical  applications, the construction of gene regulatory networks using expression profiling of genes may involve jointly analyzing gene expression levels alongside categorical phenotypes. Similarly, diagnostic data in many medical applications may contain continuous measurements such as blood pressure and discrete information about disease status or pain levels.

In analyzing such data, estimating a joint multivariate graphical model spanning the various variable types is often of interest. In practice, this is sometimes done using \textit{ad hoc} pipelines and data transformations. However, in graphical modeling, since the model output is intended to be scientifically interpretable and involves statements about properties such as conditional independence between variables, the use of \textit{ad hoc} workflows without an understanding of the resulting estimation properties is arguably problematic.

There have been three main lines of work that tackle high-dimensional graphical modeling for mixed data. The earliest approach is conditional Gaussian modeling of a mix of categorical and continuous data \cite{Lauritzen96} as treated by \citet{Cheng17, Lee15}. A second approach is to employ neighborhood selection, which amounts to separate modeling of conditional distributions for each variable given all others \cite[see, e.g.][]{Chen15, Yang14, Yang19}. A third approach uses latent Gaussian models, with a key recent reference being the paper of \citet{Fan17}, who proposed a latent Gaussian copula model for mixed data. The generative structure in their work posits that the discrete data is obtained from latent continuous variables thresholded at certain (unknown) levels.  However, in \cite{Fan17}, only a mix of binary and continuous data is considered. Their setting does not allow for more general combinations (including counts or ordinal variables) as found in many real-world applications.

This third approach will be the focus of this paper, which aims to provide a simple framework for working with latent Gaussian copula models to analyze general mixed data. To do so, we combine classical ideas concerning polychoric and polyserial correlations with approaches from the high-dimensional graphical models and copula literature. As we discuss below, this provides an overall framework that is scalable, general, and straightforward from the user's point of view.

Already in the early 1900s, \citet{Pearson1900, Pearson13} worked on the foundations of these ideas in the form of the tetrachoric and biserial correlation coefficients. From these arose the maximum likelihood estimators (MLEs) for the general version of these early ideas, namely the polychoric and the polyserial correlation coefficients. One drawback of these original measures is that they have been proposed in the context of latent Gaussian variables. A richer distributional family is the nonparanormal proposed by \citet{Liu09} as a nonparametric extension to the Gaussian family. A random vector $\boldsymbol{X} \in \mathbb{R}^d$ is a member of the nonparanormal family when $f(\boldsymbol{X}) = (f_{1}(X_{1}), \dots, f_{d}(X_{d}))^{T}$ is Gaussian, where $\{f_{k}\}_{k=1}^{d}$ is a set of univariate monotone transformation functions. Moreover, if the $f_j$'s are monotone and differentiable, the nonparanormal family is equivalent to the Gaussian copula family. As the polychoric and polyserial correlation assumes that observed discrete data are generated from latent continuous variables, they adhere to a latent copula approach.

We propose two estimators of the latent correlation matrix, which can subsequently be plugged into existing precision matrix estimation routines, such as the graphical lasso (glasso) \cite{Friedman08}, CLIME \cite{Cai11}, or the graphical Dantzig selector \cite{Yuan10}. The first is appropriate under a latent Gaussian model and unifies the aforementioned MLEs. The second is more general and is applicable under the latent Gaussian copula model. Both approaches can deal with discrete variables with arbitrarily many levels. We that both estimators exhibit favorable theoretical properties and include empirical results based on real and simulated data. The main contributions of the paper are as follows:
\begin{itemize}
    \item We posit that integrating polychoric and polyserial correlations into the latent Gaussian copula framework offers an elegant, straightforward, and highly effective approach to graphical modeling for comprehensively diverse mixed data sets.
    \item We present theoretical findings on the performance of the proposed estimators, encompassing their behavior in high-dimensional scenarios. The concentration results underscore the statistical validity of the introduced procedures.
    \item We empirically examine the estimators through a series of simulations and a practical example involving real phenotyping data of mixed types sourced from the UK Biobank. Our findings illustrate the practical utility of the proposed methods, demonstrating that their performance often closely aligns with an oracle model granted access to true latent data.
\end{itemize}

Our proposed procedure provides users with a method for conducting statistically sound graphical modeling of mixed data that is both straightforward to implement and carries no more overhead than conventional high-dimensional Gaussian graphical modeling approaches. Our procedure requires no manual specification of variable-type-specific model components, such as bridge functions.

The remainder of this paper is organized as follows. In Sections \ref{sec::gaussian} and \ref{sec::nonparanormal}, we present the estimators based on polychoric and polyserial correlations, including theoretical guarantees in terms of concentration inequalities. In Section \ref{sec::numerical_results}, we describe the experimental setup used to test the proposed approaches on simulated data together with the results themselves. %Section \ref{sec::empirical_application} showcases an illustrative empirical application using real data from the UK Biobank.
We conclude with a summary of our findings in Section \ref{sec::conclusions} and point towards our \texttt{R} package \pkg{hume}, providing users with a convenient implementation of the methods developed in this study.

