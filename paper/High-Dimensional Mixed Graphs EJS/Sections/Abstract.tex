
%Graphical models are an important tool in biostatistics and biomedical machine learning.
%Estimators of graphical or network structure are widely used in exploring relationships between variables in complex, multivariate data.
\noindent Graphical models are an important tool in exploring relationships between variables in complex, multivariate data. Methods for learning such graphical models are well-developed in the case where all variables are either continuous or discrete, including in high dimensions. However, in many applications, data span variables of different types (e.g., continuous, count, binary, ordinal, etc.), whose principled joint analysis is nontrivial. Latent Gaussian copula models, in which all variables are modeled as transformations of underlying jointly Gaussian variables, represent a useful approach. Recent advances have shown how the binary-continuous case can be tackled, but the general mixed variable type regime remains challenging. In this work, we make the simple but useful observation that classical ideas concerning polychoric and polyserial correlations can be leveraged in a latent Gaussian copula framework. Building on this observation, we propose a flexible and scalable methodology for data with variables of entirely general mixed type. We study the key properties of the approaches theoretically and empirically. %via extensive simulations as well as an illustrative application to data from the UK Biobank concerning COVID-19 risk factors.