
In order to numerically assess the accuracy of our mixed graph estimation approach we begin with a simulation study in which the estimators can be assessed in a gold-standard fashion and compared against oracles. 
\\

\noindent
{\it Simulation strategy.} 
To facilitate comparison, we follow a similar data-generating strategy to \citet{Fan17}. We split the experiments into three parts. First, we consider a binary mixed case and benchmark the performance of our approach against \citet[][Section 6.1 scenarios c) and d)]{Fan17}. Second, we generate a mix of binary-ternary-continuous data. Although \citet{Quan18} do not report any numerical results in their simulation study, we compare our approach to their extension of \citet{Fan17}. Third, from $\boldsymbol{Z}$ we generate discrete data with arbitrary numbers of levels  and compare performance with the latent continuous oracle. The results for the binary-ternary-continuous data simulations as well as a detailed description of the simulation setup can be found in Section 6 
%\ref{sec::additional_setup_results}%
of the Supplementary Materials. We set the dimension to $d = (50,250,750)$ for sample size $n = (200,200,300)$ and choose $c$ such that the number of edges is roughly equal to the dimension -- except for $d = 50$, where we allow for $200$ edges in accordance with \citet{Fan17}.       
\\

\noindent
{\it Performance metrics.} 
To assess performance, we report the mean estimation error $\Norm{\hat{\boldsymbol{\Omega}} - \boldsymbol\Omega^*}_F$ 
as evaluated by the Frobenius norm.
%Here $\hat{\Omega}$ is chosen by minimizing the eBIC according to the procedure outlined in Section \ref{sec:precision_matrix} with $\theta = 0.1$ for the low and medium, $\theta = 0.5$ for the high dimensional graphs.
Furthermore, we consider graph recovery metrics. To this end, we define the number of true positives $\text{TP}(\lambda)$ and false positives $\text{FP}(\lambda)$ depending on the \textit{glasso path} as the number of nonzero lower off-diagonal elements that agree both in $\boldsymbol\Omega^*$ and $\hat{\boldsymbol\Omega}$ and the number of nonzero lower off-diagonal elements in $\hat{\boldsymbol\Omega}$ that are actually zero in $\boldsymbol\Omega^*$, respectively. The true positive rate $\text{TPR}(\lambda)$ and the false positive rate $\text{FPR}(\lambda)$ are defined as $\text{TPR} = \frac{\text{TP}(\lambda)}{\abs{E}} $ and $\text{FPR} = \frac{\text{FP}(\lambda)}{d(d-1)/2 - \abs{E}}$, respectively. Lastly, we consider the area under the curve (AUC) where a value of $0.5$ corresponds to random guessing of the presence of an edge and a value of $1$ corresponds to perfect error-free recovery of the underlying latent graph (in the rank sense of ROC analysis).
%Note, that neither TPR nor FPR or AUC depend on the eBIC but rather on the set of tuning parameters $\lambda$ in Eq. \eqref{glasso}.     

