To numerically assess the accuracy of our mixed graph estimation approach, we commence with a simulation study in which the estimators can be rigorously evaluated in a gold-standard fashion and compared against oracles.
\paragraph{Simulation strategy.}
To enable a comprehensive comparison, we adopt a data-generating strategy akin to that of \cite{Fan17}. Our experiments unfold in three parts. Initially, we delve into a binary mixed case, benchmarking our approach against scenarios c) and d) in Section 6.1 of \cite{Fan17}. Subsequently, we generate a combination of binary-ternary-continuous data. Despite the absence of numerical results in the simulation study by \citet{Quan18}, we compare our approach with their extension of \citet{Fan17}. In the third part, we generate discrete data with varying levels from $\mathbf{Z}$ and assess performance against\todo{the ensemble estimator proposed by \cite{Feng19}} the latent continuous oracle.

For detailed results and a comprehensive description of the simulation setup for binary-ternary-continuous data simulations, refer to Section 6 of the Supplementary Materials. We set the dimension to $d = (50, 250, 750)$ for sample size $n = (200, 200, 300)$ and choose $c$ such that the number of edges aligns roughly with the dimension -- except for $d = 50$, where we allow for $200$ edges following \citet{Fan17}.

\paragraph{Performance metrics.}
    % To assess performance, we report the mean estimation error $\Norm{\hat{\mathbf{\Omega}} - \mathbf\Omega^*}_F$ as evaluated by the Frobenius norm. Furthermore, we consider graph recovery metrics. To this end, we define the number of true positives $\text{TP}(\lambda)$ and false positives $\text{FP}(\lambda)$ depending on the \textit{glasso path} as the number of nonzero lower off-diagonal elements that agree both in $\mathbf\Omega^*$ and $\hat{\mathbf\Omega}$ and the number of nonzero lower off-diagonal elements in $\hat{\mathbf\Omega}$ that are actually zero in $\mathbf\Omega^*$, respectively. The true positive rate $\text{TPR}(\lambda)$ and the false positive rate $\text{FPR}(\lambda)$ are defined as $\text{TPR} = \frac{\text{TP}(\lambda)}{\abs{E}} $ and $\text{FPR} = \frac{\text{FP}(\lambda)}{d(d-1)/2 - \abs{E}}$, respectively. Lastly, we consider the area under the curve (AUC) where a value of $0.5$ corresponds to random guessing of the presence of an edge and a value of $1$ corresponds to perfect error-free recovery of the underlying latent graph (in the rank sense of ROC analysis).
To evaluate performance, we report the mean estimation error $\Norm{\hat{\mathbf{\Omega}} - \mathbf\Omega^*}_F$ using the Frobenius norm. Additionally, we employ graph recovery metrics. For this purpose, we introduce the number of true positives $\text{TP}(\lambda)$ and false positives $\text{FP}(\lambda)$ based on the \textit{glasso path}. $\text{TP}(\lambda)$ represents the count of non-zero lower off-diagonal elements that are consistent both in $\mathbf\Omega^*$ and $\hat{\mathbf\Omega}$, while $\text{FP}(\lambda)$ denotes the count of non-zero lower off-diagonal elements in $\hat{\mathbf\Omega}$ that are actually zero in $\mathbf\Omega^*$.

The true positive rate $\text{TPR}(\lambda)$ and false positive rate $\text{FPR}(\lambda)$ are defined as $\text{TPR} = \frac{\text{TP}(\lambda)}{\abs{E}}$ and $\text{FPR} = \frac{\text{FP}(\lambda)}{d(d-1)/2 - \abs{E}}$, respectively.\todo{Adapt to table that is ending up in the paper.} Finally, we consider the area under the curve (AUC), where a value of $0.5$ corresponds to random guessing of edge presence and a value of $1$ indicates perfect error-free recovery of the underlying latent graph (in the rank sense of ROC analysis).